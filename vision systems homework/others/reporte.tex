\documentclass[spanish,10pt,letterpaper, twocolumn]{article}
%\usepackage[utf8]{inputenc}
\usepackage{babel}
\usepackage{url}
\usepackage[lmargin=1.5in, rmargin=1in,tmargin=1in, bmargin=1in]{geometry}
\usepackage{amsmath}
\usepackage{amsfonts}
\usepackage{amssymb}
\usepackage{bm}
\usepackage{graphicx}

\title{Tarea 01 -Modelo de control, sistema de visi\'on y planning-}
\author{Rob\'otica, Ingenier\'ia Mecatr\'onica\\
Instituto de Ingenier\'ia y Tecnolog\'ia, UACJ\\
Catedr\'atico: Dr. Edgar A. Martinez\\
%%%Nombre(s) Alumno(s)
\textbf{Jos\'e F\'elix Velazco Velazco}\\ 
\url{al140621@uacj.mx}
}
\date{\today}

\providecommand{\e}[1]{\ensuremath{\times 10^{#1}}}

\begin{document}
\maketitle
\section{Abstract}
Pudin
\section{Modelo de generacion de trayectorias}
Para la trayectoria a seguir delrobot, se opt\'o por tomar una funci\'on $y(x)=\sin(x)$. La trayectoria a seguir debe de tener una amplitud de -0.8 a 0.8, y que se complete una onda completa empezando de 0 a 2, por lo que la funci\'on $y(x)$ como se ve en la ecuaci\'on \eqref{eq:eq1}.

\begin{equation}
{
	\label{eq:eq1}
	y(x)=0.8\cdot \sin\left( \frac{2\pi}{x^{max}}\cdot x\right) 
}
\end{equation}

Donde $x^{max}$ ser\'a la distancia m\'axima de la onda, en este caso es igual a 2.
Se obtuvieron 1000 valores de $x$, de 0 a 2 con incrementos de 0.002\\
y con estos valores se calcularon los valores de $y(x)$ y teniendo estos valores en una tabla, se obtuvieron $\dot{x}$ y $\dot{y}$, mediante derivadas num\'ericas, obteniendo las ecuaciones \eqref{eq:eq2} y \eqref{eq:eq3} para obtener nuevamente 1000 valores

\begin{equation}
	\label{eq:eq2}
	\dot{x}=\frac{dx}{dt}=\frac{x_t-x_{(t-1)}}{\Delta t}
\end{equation} 

\begin{equation}
	\label{eq:eq3}
	\dot{y}=\frac{dy}{dt}=\frac{y_t-y_{(t-1)}}{\Delta t}
\end{equation} 

Con estos valores, se obtuvieron los valores de $\dot{s}$ y $\theta$ con las ecuaciones \eqref{eq:eq4} y \eqref{eq:eq5}.

\begin{equation}
	\label{eq:eq4}
	\dot{s}=\sqrt{\dot{x}^2+\dot{y}^2}
\end{equation} 

\begin{equation}
	\label{eq:eq5}
	\theta=\arctan\left(\frac{\dot{y}}{\dot{x}}\right)	
\end{equation}

Al obtener los valores de $\dot{s}$ y $\theta$, estos valores se meter\'an en una regresi\'on de grado 7, 

Se sabe que $y=A \cdot x$, al despejar x, obtendremos la soluci\'on inversa $x=A^{-1} \cdot y$. En la ecuaci\'on \eqref{eq:eq6} se puede ver en su forma matricial, donde $y_i$ se sustituir\'a por $\dot{s}_i$ y $\theta_i$. 

\begin{equation}
\label{eq:eq6}
\begin{bmatrix}
	a_0 \\
	a_1 \\
	\vdots \\
	a_7
\end{bmatrix}
	=
\begin{bmatrix}
	N & \Sigma t  & \cdots & \Sigma t^7\\
	\Sigma t & \Sigma t^2  & \cdots & \Sigma t^8\\
	\vdots & \vdots & \ddots & \vdots \\
	\Sigma t^7 & \Sigma t^8 & \cdots & \Sigma t^{14}
\end{bmatrix} ^{-1}
\cdot
\begin{bmatrix}
	\Sigma(y_i)\\
	\Sigma(t_iy_i)\\
	\vdots \\
	\Sigma(t_i^7y_i)
\end{bmatrix}
\end{equation}

Esta operaci\'on nos dar\'a los polinomios de $v(t)=\dot{s}(t)$ y $\theta(t)$.

\begin{equation}
	\label{eq:eq7}
	\begin{matrix}
	v(t)=-4.19\e{-5}+0.02t-0.005t^2\\
	+5.88\e{-4}t^3-3.17\e{-5}t^4\\
	+8.35\e{-7}
t^5-8.57\e{-9}t^6
	\end{matrix}
\end{equation}

\begin{equation}
	\label{eq:eq8}
	\begin{matrix}
	\theta(t)=-9.52\e{-5}+0.06t-0.015t^2\\
	+0.0013t^3-5.89\e{-5}t^4+1.41\e{-6}t^5\\
	-1.43345\e{-8}t^6
	\end{matrix}
\end{equation}

Se tiene la velocidad lineal $v(t)$ como se ve en la ecuaci\'on \eqref{eq:eq7}. Para la velocidad angular, solo derivamos el polinomio $\theta(t)$, que se ve en la ecuaci\'on \eqref{eq:eq8}, para obtener finalmente $\omega (t)$ 
 
\begin{equation}
	\label{eq:eq9}
	\begin{matrix}
		\omega (t)=0.06-0.029t+0.004t^2-2.4\e{-4} t^3 \\
		 +7.076\e{-6}t^4-8.6\e{-8}t^5
	\end{matrix}
\end{equation}



%%%%%%%%%%%%%
%%Las secciones del reporte tendran el titulo mas idoneo posible. No existe un numero para el total de secciones.
\section{Modelo odom\'etrico}

\subsection{Od\'ometros}
Tomando en cuenta que el amigobot sensa mediante encoders, entonces se obtiene el siguiente modelo

\begin{equation}
	\label{eq:eq10}
	s=\frac{2\pi r}{R}* \eta
\end{equation}

Con una derivada num\'erica sacamos las velocidades de cada llanta, as\'i como se ve en la ecuaci\'on \eqref{eq:eq12}

\begin{equation*}
	\label{eq:eq11}
	v_D=\frac{2\pi r}{R \Delta t}(\eta_{R2}-\eta_{R1})
\end{equation*}

\begin{equation}
	\label{eq:eq12}
	v_L=\frac{2\pi r}{R \Delta t}(\eta_{L2}-\eta_{L1})
\end{equation}

La velocidad lineal del amigobot ser\'ia el promedio de las velocidades de las llantas. Y a partir de la ecuaci\'on \eqref{eq:eq13}, podemos inferir un modelo en donde se vean implicados las velocidades angulares de las llantas, como en la ecuaci\'on \eqref{eq:eq14}.


\begin{equation}
	\label{eq:eq13}
	v_t=\frac{V_R+V_L}{2}
\end{equation}

\begin{equation*}
	v_{(R,L)}=r \varphi_{(R,L)}
\end{equation*}

\begin{equation*}
	\therefore v_t=\frac{r\dot{\varphi}_R+r\dot{\varphi}_L}{2}
\end{equation*}

\begin{equation}
	\label{eq:eq14}
	v_t(\dot{\varphi}_R,\dot{\varphi}_L)=\frac{r}{2}(\dot{\varphi}_R+\dot{\varphi}_L)
\end{equation}

Ahora obtendremos el modelo de velocidad angular en funci\'on de las velocidades angulares de las llantas, asumiendo que 

\begin{equation}
	\label{eq:eq15}
	\hat{v}_t=v_R-v_L
\end{equation}

Entonces, tendremos que

\begin{equation}
	\label{eq:eq16}
	\omega_t=\frac{\hat{v}_t}{\frac{b}{2}}=\frac{2}{b}\hat{v}_t
\end{equation}

\begin{equation}
	\label{eq:eq17}
	\omega_t(\dot{\varphi}_R,\dot{\varphi}_L)=\frac{2r}{b}(\dot{\varphi}_R-\dot{\varphi}_L)
\end{equation}

Y con la ecuaci\'on \eqref{eq:eq14} y \eqref{eq:eq17}, ser\'a nuestra ley de control.

\subsection{Cinem\'atica}

A partir de nuestra ley de control, vista en el tema pasado, podemos describirlo de manera matricial. 

\begin{equation}
	\label{eq:eq18}
	\begin{bmatrix}
		v_t \\ \\
		w_t
	\end{bmatrix}
	=
	\begin{bmatrix}
		\frac{r}{2} & \frac{r}{2} \\ \\
		\frac{2r}{b} & \frac{2r}{b}
	\end{bmatrix}
	\cdot
	\begin{bmatrix}
		\dot{\varphi}_R \\ \\
		\dot{\varphi}_L
	\end{bmatrix}
\end{equation} 

Para obtener nuestro vector de control, as\'i como se ve en la ecuaci\'on \eqref{eq:eq18}, tendremos que nuestra matriz de restricci\'on cinem\'atica lo multiplicaremos por nuestro vector de variables de control independiente, las cuales son las velocidades angulares de las llantas. 

Esto se hace cuando se conocen las velocidades angulares de las llantas, pero se desconoce el valor las velocidades $v_t$ y $\omega_t$. 

En caso de que si sepamos las velocidades $v_t$ y $\omega_t$, pero no las velocidades $\dot{\varphi}_R$ y $\dot{\varphi}_L$, entonces se opta por la soluci\'on inversa, con lo que tendriamos:

\begin{equation}
	\label{eq:eq19}
	\begin{bmatrix}
		\dot{\varphi}_R \\ \\
		\dot{\varphi}_L
	\end{bmatrix}
	=
	\begin{bmatrix}
		\frac{r}{2} & \frac{r}{2} \\ \\
		\frac{2r}{b} & \frac{2r}{b}
	\end{bmatrix}^{-1}
	\cdot
	\begin{bmatrix}
		v_t \\ \\
		w_t
	\end{bmatrix}
\end{equation} 

 
%%%%%%%%%%%%%%%%%%%%%%%%%%%%%%%%%%%%%%%%%%%%%%%%%%%%%%%%%%%%%%%%%%%


\section{Modelo de sensado visual}

\section{Resultados}


\section{nombre-de-seccion}
Las referencias bibliograficas se citan de la sig. manera \cite{clave1}, \cite{clave2, clave3}. 


%\begin{figure}
%\label{fig:grafica1}
%\centering
%\includegraphics[scale=0.5]{./nombrearchivo.pdf}
%\caption{Descripci\'on el comportamiento de la velocidad angular $\dot{\varphi}_a$.}
%\end{figure}

\section{nombre-de-seccion}
Los vectores $\mathbf{v}$ y matrices $\mathbf{A}$ se expresan de la siguiente manera.

\begin{equation}
\bm{\xi}_t =
\begin{pmatrix}
\cos(\phi) && \sin(\beta+\alpha)\\
\ln(\gamma) && {\rm e}^\theta-\tau
\end{pmatrix}
\cdot
\begin{pmatrix}
x\\
y
\end{pmatrix}
\end{equation}

\section{Conclusiones}
Qu\'e se logr\'o?, y no se logr'o?. Qu\'e fue dificil? qu\'e result\'o ser interesante/eficiente/f\'acil/dificil etc? Qu\'e se recomienda a partir de la experiencia obtenida?

\begin{thebibliography}{10}
\bibitem{clave1} Smith J., \textbf{Robotic mechanisms}, MIT, 2 Ed., 2010
\bibitem{clave2} Weird K., \textbf{Mechanical systems}, IEEE RAS, vol. 1, 2001
\bibitem{clave3} Da Vinci L., \textbf{Automata Machnines}, Wiley, 1999
\end{thebibliography}
\end{document}

